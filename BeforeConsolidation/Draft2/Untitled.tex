% Options for packages loaded elsewhere
\PassOptionsToPackage{unicode}{hyperref}
\PassOptionsToPackage{hyphens}{url}
\PassOptionsToPackage{dvipsnames,svgnames,x11names}{xcolor}
%
\documentclass[
]{article}
\title{Exposure Doesn't Pay the Bills}
\usepackage{etoolbox}
\makeatletter
\providecommand{\subtitle}[1]{% add subtitle to \maketitle
  \apptocmd{\@title}{\par {\large #1 \par}}{}{}
}
\makeatother
\subtitle{Artistic Production on Streaming Platforms Under
Algorithmically-Induced Audience Uncertainty}
\author{Aaron Graybill}
\date{22 January 2022}

\usepackage{amsmath,amssymb}
\usepackage{lmodern}
\usepackage{setspace}
\usepackage{iftex}
\ifPDFTeX
  \usepackage[T1]{fontenc}
  \usepackage[utf8]{inputenc}
  \usepackage{textcomp} % provide euro and other symbols
\else % if luatex or xetex
  \usepackage{unicode-math}
  \defaultfontfeatures{Scale=MatchLowercase}
  \defaultfontfeatures[\rmfamily]{Ligatures=TeX,Scale=1}
\fi
% Use upquote if available, for straight quotes in verbatim environments
\IfFileExists{upquote.sty}{\usepackage{upquote}}{}
\IfFileExists{microtype.sty}{% use microtype if available
  \usepackage[]{microtype}
  \UseMicrotypeSet[protrusion]{basicmath} % disable protrusion for tt fonts
}{}
\usepackage{xcolor}
\IfFileExists{xurl.sty}{\usepackage{xurl}}{} % add URL line breaks if available
\IfFileExists{bookmark.sty}{\usepackage{bookmark}}{\usepackage{hyperref}}
\hypersetup{
  pdftitle={Exposure Doesn't Pay the Bills},
  pdfauthor={Aaron Graybill},
  colorlinks=true,
  linkcolor={myRed},
  filecolor={Maroon},
  citecolor={Blue},
  urlcolor={myBlue},
  pdfcreator={LaTeX via pandoc}}
\urlstyle{same} % disable monospaced font for URLs
\usepackage[margin=1in]{geometry}
\usepackage{graphicx}
\makeatletter
\def\maxwidth{\ifdim\Gin@nat@width>\linewidth\linewidth\else\Gin@nat@width\fi}
\def\maxheight{\ifdim\Gin@nat@height>\textheight\textheight\else\Gin@nat@height\fi}
\makeatother
% Scale images if necessary, so that they will not overflow the page
% margins by default, and it is still possible to overwrite the defaults
% using explicit options in \includegraphics[width, height, ...]{}
\setkeys{Gin}{width=\maxwidth,height=\maxheight,keepaspectratio}
% Set default figure placement to htbp
\makeatletter
\def\fps@figure{htbp}
\makeatother
\setlength{\emergencystretch}{3em} % prevent overfull lines
\providecommand{\tightlist}{%
  \setlength{\itemsep}{0pt}\setlength{\parskip}{0pt}}
\setcounter{secnumdepth}{5}
\newlength{\cslhangindent}
\setlength{\cslhangindent}{1.5em}
\newlength{\csllabelwidth}
\setlength{\csllabelwidth}{3em}
\newlength{\cslentryspacingunit} % times entry-spacing
\setlength{\cslentryspacingunit}{\parskip}
\newenvironment{CSLReferences}[2] % #1 hanging-ident, #2 entry spacing
 {% don't indent paragraphs
  \setlength{\parindent}{0pt}
  % turn on hanging indent if param 1 is 1
  \ifodd #1
  \let\oldpar\par
  \def\par{\hangindent=\cslhangindent\oldpar}
  \fi
  % set entry spacing
  \setlength{\parskip}{#2\cslentryspacingunit}
 }%
 {}
\usepackage{calc}
\newcommand{\CSLBlock}[1]{#1\hfill\break}
\newcommand{\CSLLeftMargin}[1]{\parbox[t]{\csllabelwidth}{#1}}
\newcommand{\CSLRightInline}[1]{\parbox[t]{\linewidth - \csllabelwidth}{#1}\break}
\newcommand{\CSLIndent}[1]{\hspace{\cslhangindent}#1}
\usepackage{tikz}
\usepackage{pgfplots}
\let\textlozenge\relax
\usepackage{heuristica}
\usepackage[heuristica,vvarbb,bigdelims]{newtxmath}
\usepackage[T1]{fontenc}
\let\openbox\relax
\usepackage{amsthm}
\usepackage{amssymb}
\renewcommand*\oldstylenums[1]{\textosf{#1}}
\definecolor{myBlue}{HTML}{255059}
\definecolor{myRed}{HTML}{8C2730}
\definecolor{myBlack}{HTML}{13091C}
\interfootnotelinepenalty=10000
\newtheorem{prop}{Proposition}
\ifLuaTeX
  \usepackage{selnolig}  % disable illegal ligatures
\fi

\begin{document}
\maketitle

\setstretch{1.5}
\{r setup, include=FALSE\}
knitr::opts\_chunk\$set(echo=F,message=F,warning=F)

Introduction

YouTube, the video streaming giant, reported 19.7 Billion dollars in ad
revenue during 2020. Spotify, the music streaming platform, posted a
similarly giant 7.8 Billion dollars in revenue from 2020. Comparing
revenues to nominal GDP, this would put YouTube at a similar size to the
nation of Afghanistan. Revenue numbers like these cannot be ignored.

YouTube and Spotify, among their industry peers like Instagram, TikTok,
lie at the intersection of two burgeoning markets: streaming platforms
and social media. Services like YouTube, Spotify, Instagram, and TikTok,
which I will refer to as platforms, provide a rich opportunity for
economic analysis.

Platforms are fundamentally a marketplace. Content creators, firms, are
given an opportunity to connect with consumers who wish to consume their
product. However, platforms provide a marketplace unlike anything that
has been previously seen. In this way, streaming economies are two-sided
markets, markets in which both the buyer and sellers connect through a
third party. In our case, the streaming platform serves as the point of
connection. One key distinction between a physical marketplace and a
digital platform is that consumption is usually nonrivalrous. One
consumer's decision to listen to an artist's newest release on Spotify
has no negative impact on another consumer's ability to consume that
product. However, one consumer's choices do indeed have an effect on
others. We expect that a platform's algorithm measures one user's
engagement and uses that and other factors to decide whether or not to
show a product to a different consumer. Furthermore, a content creator's
inventory is limited only by the bandwidth of the platform and the
number of consumers who wish to consume their product. The number of
willing consumers is a content creator's audience, and is central to the
analysis conducted below.

Audience evolution is another key difference between a streaming economy
and traditional market structures. Of course, consumers can seek out
content directly if they know that a creator exists, but that does not
fully describe consumer-creator matching on a platform. A central
component of streaming markets is the algorithm. By algorithm, I mean a
platform's way of analyzing a release by a content creator and deciding
which consumers will be shown that release. In some markets, like an
outdoor farmer's market, the algorithm is nonexistent. Consumers are
required to self-select products they wish to purchase. However,
platforms use their algorithm to show content to consumers they may not
have initially been aware of. Algorithms are also a way of encoding a
content creator's reputation over time. An algorithm looks at a subset
of a content creator's history of releases when deciding who will see an
artists new releases. These past releases serve as a reputation even
when a consumer has not yet been exposed to a creator. A consumer might
think, ``I don't know what this video will be, but if the algorithm is
recommending it, I will probably enjoy it.''

All successful content creators must acknowledge the importance of a
platform's algorithm and adapt to its changes. A useful anecdote comes
from Stevie Wynne Levine, Chief Creative Officer of Mythical. Mythical
is a YouTube conglomerate with over 75 million subscribers across its
channels and 25 billion total views on its videos. In reference to how
their team analyzes the algorithm, Wynne Levine reveals:

It's something that we do not only every day, but three, four times a
day. We analyze each video that we've put out for that day and how we
can improve impressions and views. So definitely not something that we
set it and forget it, because it's constantly evolving. (para. 12)

The above discussion reveals that content creators have a unique
economic problem that is worthy of study. Recent literature has focused
on the optimal behavior of consumer and platforms, but content creators
have been mostly omitted in the literature of streaming platforms. Work
that has tackled artist's optimal behavior has primarily considered fine
art in non-digital markets. The analysis below explores whether or not
streaming platforms create conditions that mirror fine art markets by
concentrating revenue in a small minority of the total number of
producers.

In this paper, I will explore how content creators must balance the
inherent quantity-quality trade-off present in making content for
platform. A higher number of releases offers a content creator more
opportunities for the random component of the algorithm to make their
content viral. However, investing in greater quality encourages
consumers who do know about an artist's content to stream that content
more. Again using YouTube as an example, their ``Creator Academy'' which
is intended to teach content creators how to make more effective
content, acknowledges but does not provide clear guidance on this
quality-quantity tradeoff. Simon Whistler, a popular content creator
hired to present these videos says that ``\ldots{} if you have more
videos out there, chances are your watch time overall is going to be
higher''
\protect\hyperlink{ref-youtubecreatorsWhatIdealVideo2018}{YouTube
Creators}
(\protect\hyperlink{ref-youtubecreatorsWhatIdealVideo2018}{2018}). Watch
time, in this context, is the cumulative time that viewers have watched
your content divided by the number of unique viewers and is an important
predictor of algorithmic success

I construct a model of artist/content creator behavior in which artists
must decide a quantity and quality of art to release in every period.
Their quantity choice affords them a number of chances at algorithmic
exposure. Additionally, the algorithm will also factor in how many times
past audience members consumed an artist's work when deciding audience
size. An artist's underlying talent (ability to produce the same
quality/quantity at a lower cost) will govern optimal production, and I
will show how the distribution of talent relates to the distribution of
revenue in successful artists.

I will show that artists early in their career, where new audience
members make up a large share of their total audience, are more likely
to produce low quality content, and then switch to a low-volume,
high-quality strategy later in their career. The interaction of artist
decisions in the streaming economy and algorithmic uncertainty is novel
to this paper.

The remainder of this paper is broken into (SOME NUMBER) of sections. In
the literature review I discuss how related literature has informed the
expectations of the model. In the model section, I hypothesize results
based on the literature, and present the primary model used in this
analysis. In the (REST OF SECTIONS GO HERE)\ldots{}

Literature Review

In this paper I will examine how an artist's production decisions and
reputation are influenced by algorithmically uncertain audience size in
the digital streaming economy. This topic is tethered to multiple lines
of research each of which informs the construction of the model below. I
discuss connections to the branding literature, the superstar
literature, and the novel streaming literature.

The branding literature provides tools to analyze how the information
that a firm communicated, its brand, affects consumers' decisions. In
this analysis, firms must build a reputation even if that doesn't take
the form of a traditional brand, logos, typography, and other factors.
The superstar literature analyzes how market concentration can develop,
particularly in art markets. I will analyze how the introduction of an
algorithm influences the distribution of talent in an art market using
the superstar literature as a pre-digitization baseline. Finally, the
streaming literature models the incentives and optimal behavior of
streaming platforms and end users. The model below supplements the
streaming literature by analyzing the artist's problem taking the
properties of streaming platform and the end user as given.

Branding Literature

An artist concerned with growing their audience faces many similar
incentives to a new firm trying to develop a brand. One can view a brand
as a set of signals about the quality of a firm's product. Branding
becomes an important consideration when consumers do not have ex ante
knowledge about the quality of the product they are purchasing. As such,
consumers rely on the signals presented by a firm's brand to inform
their consumption decisions. The branding literature begins with the
signalling literature pioneered in
\protect\hyperlink{ref-spenceJobMarketSignaling1973}{Spence}
(\protect\hyperlink{ref-spenceJobMarketSignaling1973}{1973}). In this
paper, Spence finds that observable characteristics, both impactful and
superficial, can have substantial effects on the hiring decisions of a
potential employer. More directly,
\protect\hyperlink{ref-kleinRoleMarketForces1981}{Klein \& Leffler}
(\protect\hyperlink{ref-kleinRoleMarketForces1981}{1981}) presents a
simple model that identifies key characteristics that a market must
possess in order for firms to invest in branding and selling a high
quality product. Central to their model is consumer reputation
formation. They propose a rather draconian baseline in which consumers'
trust can never be regained upon a firm choosing to deceive them. A key
finding is that with sufficient differentiation between high and low
quality products, some firms may choose to invest in their reputation
into perpetuity. This provides evidence that even when a firm's brand
offers no intrinsic consumer utility, consumers benefit enough from the
information of a brand, that it is worthwhile for both the firm and the
consumer to invest in the more expensive branding.
\protect\hyperlink{ref-shapiroPremiumsHighQuality1983}{Shapiro}
(\protect\hyperlink{ref-shapiroPremiumsHighQuality1983}{1983})
generalizes the model proposed by Klein and Leffler to a case where
reputation can exist on a continuum of values and can evolve in a less
austere manner. Shapiro confirms the results of Klein and Leffler while
expanding on the fragility of branding. Shapiro finds that even when a
firm is able to charge more for a branded product than an unbranded
alternative, this premium is fleeting and sensitive to changes in
consumer preferences.

The application of the branding literature to the problem posed in this
paper lies in the way that streaming platforms' algorithms reveal
information and content to consumers. Generally streaming platforms
allow users to choose their own content, but many platforms like
YouTube, Spotify, and Apple Music also algorithmically provide content
based on the consumer's history of the content they have consumed and
how they have consumed it (number of times, shares, etc.). When this
content is previously unknown to the consumer, we can view the platform
as relying on its branding to present desirable content to the user. I
will extend the branding literature by exploring the context in which
reputation is subject to uncontrollable shocks that can positively or
negatively influence next period's reputation.

Superstar Literature

A related strand of literature to branding is the superstar literature
which lies more towards cultural economics. Rosen, the preeminent author
the superstar literature, characterizes a superstar market as having a:
``relatively small numbers of people {[}who{]} earn enormous amounts of
money and dominate the activities in which they engage,''
(\protect\hyperlink{ref-rosenEconomicsSuperstars1981a}{Rosen, 1981, p.
845}). (LET ME KNOW IF THAT CITATION IS DONE WRONG) In his paper, he
presents a model where consumers can perfectly observe talent ex ante.
In this framework, Rosen finds that small increases in underlying
artistic talent can have disproportionately large increases in resultant
revenue in market equilibrium. This leads to revenues being concentrated
among only a few artists. Art market concentration is further explored
in \protect\hyperlink{ref-macdonaldEconomicsRisingStars1988}{MacDonald}
(\protect\hyperlink{ref-macdonaldEconomicsRisingStars1988}{1988}) which
takes an alternate modeling approach and introduces uncertainty in the
talent of an artist. This uncertainty is market-wide where neither the
consumer nor the artist knows their talent until they have performed.
This model also intersects with the branding literature because it
explores how an artist's perceived talent evolves over multiple periods
based on the quality of their performance.

The superstar literature is important to this analysis because it
emphasizes the artist's production decisions and how they affect market
revenues. As I will discuss later, much of the recent literature on
streaming economies has focused on the streaming platform and the end
consumers, so the superstar literature gives a more nuanced view of how
artistic production can be modeled and optimized. However, I add to the
superstar literature by modernizing the analysis and seeing if the same
patterns emerge under a digitally-based economy. In particular, I will
provide further insight into how the distribution of underlying talent
may or may not be reflected in the realized distribution of talent in
successful artists.

Streaming Literature

The third and most closely related field of study is the streaming
literature. The advent of streaming has garnered considerable attention
from both theoretical and empirical angles. Streaming platforms can take
many forms, but I will follow the characterization given in
\protect\hyperlink{ref-thomesEconomicAnalysisOnline2013}{Thomes}
(\protect\hyperlink{ref-thomesEconomicAnalysisOnline2013}{2013}). He
characterizes a streaming economy as an internet-based two-sided media
marketplace. The streaming platform is in the middle of this two sided
marketplace. The first side of the marketplace is the streaming platform
accepting media from content creators. The other side of this
marketplace is the streaming service delivering this content to
consumers. Usually money is changed hands on both sides of this market.
Researchers have focused on various aspects of the streaming economy as
it has evolved. I will begin by giving an overview of some of the key
topics addressed by theoretical papers. After discussing theory, I will
mention some relevant empirical studies that analyze how digitization
has affected the distribution of artist popularity---a central question
of the research at hand.

Early papers in the streaming literature investigate how streaming may
curb pirating, the (usually free) illegal download of unlicensed music,
like on Napster. One such early paper is
\protect\hyperlink{ref-thomesEconomicAnalysisOnline2013}{Thomes}
(\protect\hyperlink{ref-thomesEconomicAnalysisOnline2013}{2013}) which
takes the perspective of the music streaming platform when deciding how
to price its paid subscription service relative to its free-with-ads
alternative. This paper takes the artist's behavior as given, and does
not tackle the conditions under which artists will choose to produce
content for the platform.

Another more recent paper that takes the streaming platform's
perspective is
\protect\hyperlink{ref-benderAttractingArtistsMusic2021}{Bender et al.}
(\protect\hyperlink{ref-benderAttractingArtistsMusic2021}{2021}). This
paper analyzes competition between permanent digital MP3 sales and
streaming platforms. It analyzes consumer demand, and how the platform
should optimally set its royalty to attract artists to the platform. The
authors find that it is the most popular artists that might choose to
hold out and sell their work only via permanent download, a result seen
anecdotally with the Beatles who kept their music off of streaming
services for a famously long time. The analysis below examines how
artists should optimally produce once they have already committed to
making content for a streaming platform.

\protect\hyperlink{ref-hillerRiseStreamingMusic2017}{Hiller \& Walter}
(\protect\hyperlink{ref-hillerRiseStreamingMusic2017}{2017})
incorporates elements from both
\protect\hyperlink{ref-thomesEconomicAnalysisOnline2013}{Thomes}
(\protect\hyperlink{ref-thomesEconomicAnalysisOnline2013}{2013}) and
\protect\hyperlink{ref-benderAttractingArtistsMusic2021}{Bender et al.}
(\protect\hyperlink{ref-benderAttractingArtistsMusic2021}{2021}) by
modelling an economy with digital purchases in addition to free and paid
subscriptions on a streaming service. The authors investigate the
artist's decision to produce one high-quality piece of art versus
multiple comparatively lower-quality pieces. The authors find that the
streaming economy fosters an environment in which profit-maximizing
artists focus on generating high-quality singles instead of
lower-quality albums. I will examine this quantity-quality trade-off
will below, but with the key difference of uncertain audience size. The
proceeding analysis acknowledges the fact that releasing more content
may increase an artist's chance of reaching a larger audience.

Another driving question of this research pertains to the debate of
whether or not streaming platforms have created a ``long tail'' of
products. I will now summarize some empirical work that analyzes this
topic.

The term long tail was popularized in
\protect\hyperlink{ref-andersonLongTailWhy2006}{Anderson}
(\protect\hyperlink{ref-andersonLongTailWhy2006}{2006}). The principle
of the long tail is that digitization of commerce allows consumer's
access to a wider variety of products---far more than a brick and mortar
store could every stock. The greater variety of products allows
consumers to pinpoint the product that best suits their preferences. A
diverse product set results in many products having a small number
consumers. As such, the distribution of consumers per product should
have much longer tail than pre-digitization. The streaming economy is an
excellent example of where this long tail might exist as Spotify can
stock hundreds of times more songs into a server than a vinyl record
store could ever stock in-store. The long tail hypothesis would suggest
a diffusion of revenues across many artists. However,
\protect\hyperlink{ref-rosenEconomicsSuperstars1981a}{Rosen}
(\protect\hyperlink{ref-rosenEconomicsSuperstars1981a}{1981}), finds
that revenue should be concentrated among only a few artists. These
factors are not necessarily at odds, we may have most of the revenue
while still having a long tail, but this paper will examine whether or
not we have a long thick tail, or an initial bump with a long thin tail
afterwards.

That being said, opinions are mixed as to whether or not the long tail
hypothesis holds empirically.
\protect\hyperlink{ref-elberseShouldYouInvest2008}{Elberse}
(\protect\hyperlink{ref-elberseShouldYouInvest2008}{2008}) initially
pushed back against the idea of a long tail. She cites data from
Quickflix (a now-defunct Australian pay-per-view movie streaming
platform) which showed that a small number of DVDs comprised a large
portion of sales. The proceeding analysis in this paper will model a
different type of streaming platform, one in which customers have
unlimited access to content either for free or for a monthly
subscription fee.

A more recent alternate perspective on the long tail,
\protect\hyperlink{ref-aguiarQualityPredictabilityWelfare2018}{Aguiar \&
Waldfogel}
(\protect\hyperlink{ref-aguiarQualityPredictabilityWelfare2018}{2018}),
posits digitization substantially lowers the entry cost of new firms.
Lower entry costs allow firms with lower expected profit into the
market, increasing the diversity of sellers. The authors robustly show
that consumers value having a variety of producers when it is difficult
to forecast an artist's talent before purchasing. They argue that the
value of variety provides further justification that there exists a long
tail of producers. I will examine whether or not producers naturally
form a long tail even under uncertainty about the number of consumers
that they will be able to reach with their product.

This Paper's Contribution to the Literature

The preceding discussion reveals a few key unexplored area that this
paper will investigate. One, previous literature has focused on behavior
of the streaming platform and the end-user. This analysis contributes to
the comparatively-understudied role of the artist in the streaming
economy. Two, this paper will endogenize audience development, a
similarity to the branding and superstar literature not yet applied to
the streaming literature. Third, this paper will contribute the debate
of the long tail in the streaming literature by exploring the
equilibrium distribution of artist talent. Finally, This paper will also
contribute to the presently-unexplored role of unpredictable algorithms
on streaming platforms in shaping an artist's career and optimal
behavior.

The Model

The streaming economy has three primary actors: the end consumer, the
streaming platform, and the artist. For the purposes of this analysis, I
will assume that the behavior of the consumer and the streaming platform
are exogenous. I will limit attention to the artist who must choose a
quality, \(z_t\) and quantity, \(m_t\) of art to produce in every
period. The streaming platform's algorithm is driven by consumer
engagement last period, \(n_{t-1}\). The algorithm combines past user
engagement with the number of new releases to decide the number of new
consumers to show an artist's work to. I will assume that consumers have
no ex ante knowledge of an artist and require the algorithm to reveal an
artist to them. I will call these algorithmic revelations, impressions
and denote them \(I_t\). Once a consumer has received an impression,
their number of streams will depend on the quality of the art that the
artist released in that period. If a consumer has received an
impression, they are under no obligation to stream, the number of
streams is fully determined by the quality of the art. Impressions only
serve as a way for a consumer to be made aware of an artist's portfolio.
After the consumer decides to stream, the algorithm observes the streams
per impression and uses this to decide how many times to show the
artist's work next period.

The artist earns a royalty every time a song is streamed, and the artist
tries to maximize this discounted flow of royalties. The trade-off
between quantity of releases and the quality of those releases will be
central to their maximization problem. Increases to quality ensure that
once a consumer receives an artist's work, they will consume that
product more. Increasing quality also has the inter-temporal benefit of
encouraging next period's algorithm to show the art to more consumers.
On the other hand, the artist can increase their quantity at the expense
of quality. The algorithm will have more pieces to show to consumers
(increasing the probability that a given consumer discovers an artist),
but it also leaves audience size more up to the random component of the
algorithm. I will not model the algorithm in any more detail than the
components above. The algorithm, in this analysis, will serve only to
link the number of impressions this period to the number of impressions
as given. Further analysis is required on a streaming platform's optimal
algorithm construction, but that is outside the interests of this paper.

Hypotheses

Before describing the model in detail, I will first lay out hypothesized
results based on the literature presented above. The first hypothesis
pertains to the ``long tail'' theory described in
\protect\hyperlink{ref-aguiarQualityPredictabilityWelfare2018}{Aguiar \&
Waldfogel}
(\protect\hyperlink{ref-aguiarQualityPredictabilityWelfare2018}{2018}).
Following \protect\hyperlink{ref-rosenEconomicsSuperstars1981a}{Rosen}
(\protect\hyperlink{ref-rosenEconomicsSuperstars1981a}{1981}), I will
measure the long tail of outcomes by examining the convexity of the
expected profit function in an artist's underlying talent. I suspect
that the imperfect information present on streaming platforms will make
it harder for consumers to find and stay with artists that best fit
their preference which may weaken the market power of the most talented
artists. I pose the following:

Hypothesis 1: Unpredictability of a streaming platform's content
matching algorithm will decrease the convexity of profits in talent.

The second hypothesis relates to audience development. Similar to the
work of \protect\hyperlink{ref-benderAttractingArtistsMusic2021}{Bender
et al.}
(\protect\hyperlink{ref-benderAttractingArtistsMusic2021}{2021}), I will
examine how the choices of established artists differ from artists with
smaller starting audiences. In a one period model, smaller artists do
not carry a large audience from the previous period, so a high quality
product does not get shown to enough consumers to justify a high quality
strategy. As such, I propose:

Hypothesis 2: Artists with smaller initial audiences are more likely to
choose a low-quality, high-quantity strategy relative to established
artists.

The final hypothesis pertains to how the previous hypothesis is affected
by time. When there are multiple periods, a relatively new artist can
rely on producing a high quality product knowing that the future
benefits of high quality products will outweigh the short lived and
uncertain benefits of a low quality product. I propose:

Hypothesis 3: Artists will prioritize quality more in a multi-period
setting than they will in a one-period environment.

Consumer Behavior

I assume there is an infinitely large market of consumers with identical
preferences on a streaming platform. This assumption does not fully
depict the consumer-base of a streaming platform. However, for an
emerging artist in an established genre, this assumption is more
realistic. To a new artist, the number of potentially-accessible
consumers can be almost limitless. Further, if an artist enters an
existing genre, there is a clear sense of what is successful across the
entire market, so there is some observable homogeneity in consumer
preferences.

Ex ante, consumers have no knowledge of a given artist and require the
algorithm to reveal an artist to them. Upon receiving an impression of
an artist, the consumer gains knowledge of all of an artist's work from
that period, not just the piece they were exposed to. I will assume that
after an impression a consumer will stream the artist's work \(n(z)\)
times, where \(z\) is the quality of the art. I will assume that
\(n_z>0\) and \(n_{zz}<0\), so increases to quality always increase
demand, but at a lessening rate. A notable assumption in the above
modelling decision is that the quantity of releases has no influence on
the amount of streams by a given consumer. Consumers see the quality and
might choose to stream one song all \(n\) times, or they might spread
their consumption across multiple pieces of art.

Artist Behavior

The artist's problem is to maximize their discounted stream of profits
over multiple periods. The artist generates revenues from per-stream
royalties and faces a cost accordinig to their quality and quantity. In
particular, I will assume that each stream earns the artist an amount
\(r\), so the total revenue in each period is the total number of
streams time \(r\). The artist chooses to produce \(m\) pieces each of
which at the same quality \(z\). I will assume that the artist's
combination of \(m\) and \(z\) also incur a cost \(C(m,z;\kappa)\) where
\(\kappa\) is the artist's underlying talent that makes production
easier. I will impose some standard assumptions on \(C\). Namely, I will
assume \(C_m>0,C_z>0,C_{mm}>0,C_{zz}>0,C_{z\kappa}<0,C_{m\kappa}<0\).
The assumptions say that increases to quality or quantity also increase
cost, and additional units of quantity of quality are more costly than
the previous. The cross partial derivatives say that increasing talent
decrease the marginal cost in each of the inputs. In order for
\(\kappa\) to be meaningful as a talent parameter, we should require
that at every input, an additional unit of talent makes the next unit of
production less costly. I will further assume \(C_\kappa<0\) and
\(C_{\kappa\kappa}>0\), so increases to underlying talent lower costs,
but additional increases to talent are less and less impactful. The only
sign without an immediate sign choice is the quantity-quality cross
partial derivative, \(C_{mz}=C_{zm}\). A positive cross partial
derivative says that a one unit increase in quality increases the
marginal cost of an additional unit of quantity. We can also interpret
this as a one unit increase in quantity increases the marginal cost of
an additional unit of quality. More broadly, the sign of this partial
derivative determines whether or not the inputs are complementary. For
this analysis, I will assume that \(C_{mz}\geq 0\). This assumption is
reasonable considering that an artist only has finite time in every
period, so every minute they spend on using more of one input, decreases
the time they have for the other.

Audience Evolution and Algorithmic Behavior

First, a distinction between audience and streams. For the purposes of
this analysis, an artist's audience at time \(t\), \(A_t\), if the
number of consumers who are aware of an artists work. Again, these
audience members are under no obligation to stream, but they will be
able to observe an artists quality to inform their consumption
decisions.

I impose three heuristics on how an artist's audience evolves over time.
First, some proportion of last period's audience should be retained
between periods. Second, a streaming platform should have an algorithm
to determine the number of new audience members that an artist has.
Finally, there is some unforecastable random noise present in how the
algorithm behaves, at least in the eyes of the artist.

I will first assume that every period, a fraction \(\delta\) of the
audience ``forgets'' about an artist and needs reimpression in order to
consume an artist's work again. As such, the share of last period's
audience that endures is \((1-\delta)\). We can think of \(1-\delta\) as
the share of audience members who naturally remember an artist or as the
fraction of consumers who are forced to remember an artist from
reminders by the streaming platform.

Now I will characterize how new consumers can be added to an artist's
audience. I assume that the streaming platform's algorithm governs the
number of new impressions for each song released by an artist. For the
algorithm to be a meaningful tool, it should not be entirely random. It
should use some measure of engagement to dictate how many impressions in
the next period. For the purposes of this model, the algorithm will use
last period's number of streams per audience member as the measure of
engagement. By construction, at time \(t\), the algorithm will use
\(n(z_{t-1})\), so the algorithm is indirectly incorporating quality.

There are multiple ways to interpret the uncertainty in the algorithm.
One such way is to interpret a streaming platform's algorithm as an
imperfect instrument that measures talent. An alternate way to interpret
algorithmic uncertainty is in the context of producer uncertainty. In
this case, the artist understands the average effects of the algorithm,
but the streaming platform intentionally or inadvertently obfuscates
exactly how the algorithm behaves so there is always some artist
uncertainty about the true number of impressions in the next period.

\protect\hyperlink{ref-tomscottWhyYouTubeAlgorithm2017}{Tom Scott}
(\protect\hyperlink{ref-tomscottWhyYouTubeAlgorithm2017}{2017}) gives an
engaging look into why YouTube's algorithm is unpredictable. He notes
that when a streaming platform reveals information about a streaming
platform's algorithm, get-rich-quick content creators produce inferior
content that is only intended to exploit these trends. He also notes
that many algorithms are constructed using machine learning and huge
training data sets and have to many inputs to be easily understood.

An additional concern when modeling an uncertain algorithm is that
artist with higher talent may be favored by the random component of the
algorithm. However, most artists, at least for professional content
creators, can observe any information in the algorithm that might
benefit them when making their product. By definition, the most
algorithmically favored products are shown the most. As such, I will
assume that artists have ample opportunity to analyze successful content
and arbitrage away any useful information. This assumption says that
there are no frictions in obtaining information about the algorithm.
Certainly dedicated media teams have a better ability to spot
algorithmic trends than single producers, but this relationship is
outside the scope of this paper. Further analyses could explore this
relationship in more detail, but this model will assume that talent and
algorithmic uncertainty are independent.

With the aforementioned assumption, I construct the algorithm as
follows. First denote the impression algorithm for an artist's \(i\)th
art piece in period \(t\) as \(I(n_{t-1})+\varepsilon_{it}\) where
\(\varepsilon_{it}\) is a mean zero independent identically distributed
random variable. Both \(I\) and \(\varepsilon\) have units of number of
additional people in audience. Assuming the algorithm is a useful, if
imperfect, measure of quality, I will assume that \(I_n>0\). So
increases to engagement increase the expected number of impressions per
release.

For each art piece that an artist produces in the present period, they
must submit this work through the algorithm which is subject to random
noise. If the artist releases \(m\) pieces in a given period, the total
number of impressions from the audience is then given by
\(mI(n_{t-1})+\sum_{i=1}^m \varepsilon_{it}\). As such, the expected
number of impressions is simply \(mI(n_{t-1})\) with variance
\(m\textrm{Var}(\varepsilon)\). Increasing the number of releases
increases the expected audience, but equally increases the variance of
outcomes. Putting all of these pieces together, the equation of motion
for audience size at time \(t\), \(A_t\), is given by:

\begin{equation} \label{eq:eqn_of_motion}
A_t=(1-\delta)A_{t-1}+mI(n_{t-1})+\sum_{i=1}^m\varepsilon_{it}
\end{equation}

We can visualize audience development as follows:

\{r Visualize Audience Development, fig.cap=``Increases to quantity
produced increase expected audience and variance of
outcomes,''fig.align=`center'\} m=seq(0,10,.001) A\_0=4
I\_up=.75\emph{A\_0+m+m}1.1 I\_down=.75\emph{A\_0+m-m}1.1
I\_mid=.75*A\_0+m

q \textless- tibble::tibble(m,I\_up,I\_down,I\_mid)

library(ggplot2) library(latex2exp) ggplot(q)+
geom\_line(aes(m,I\_mid,color=``Expected Audience''),size=1.1)+
geom\_ribbon(aes(m,ymin=I\_down,ymax=I\_up,color=``Range of
Audience''),alpha=.2,fill=``\#211030'')+ theme\_bw()+
\#annotate(``text,'' x = 1, y = 1, label =
TeX(``\((1-\\delta)A_{t-1}\)''),family=``serif'')+
ylab(TeX(``\nAudience at time \(t\), \(A_t\)''))+ xlab(TeX(``Number of
Releases, \(m\)''))+ xlim(0,10)+ ylim(0,25)+
scale\_color\_manual(values=c(``\#8C2730,''``\#211030''))+
theme(text=element\_text(family=``serif''))+
theme(legend.title=element\_blank())+ scale\_y\_continuous(breaks =
c(3), labels = c(TeX(``\((1-\\delta)A_{t-1}\)'')))+
scale\_x\_continuous(breaks = c(0), labels = c(0))+
theme(panel.grid.major = element\_blank(), panel.grid.minor =
element\_blank(), panel.background = element\_blank(), axis.line =
element\_line(colour = ``black''))\#+ \#opts(axis.title.x =
theme\_text(vjust=-0.5))

The Artist's One-Period Maximization Problem

We can assemble the above pieces into the respective one-period revenue
maximization problem. The timing of the model is as follows, the artist
chooses \(m\) and \(z\), then the random variables in the algorithm are
realized, then consumers stream the artist work according to \(n(z)\).
In the case where there is only one period, we can interpret \(n_{t-1}\)
as the total amount of reputation that an artist has earned over the
duration of their career up to now. Similarly, \(A_{t-1}\) becomes the
entire audience that has been retained. In a one period model, objects
in the equation of motion represent a lifetime's work, not just one
period.

As such, we can express the artist's problem as:

\begin{equation} \label{eq:one_period_general}
V(m,z)=\max_{m,z}\left\{E\left[rA_tn(z)-C(m,z;\kappa)\right]  \ \textrm{s.t.} \ A_t=(1-\delta)A_{t-1}+mI(n_{t-1})+\sum_{i=1}^m\varepsilon_{it}\right\}
\end{equation}

Substituting in with \(A_t\) allows us to solve this problem as an
unconstrained maximization problem with the following first order
conditions:

\begin{equation} \label{eq:general_focs}
\begin{split}
\frac{\partial V}{\partial z}&\colon r\left[(1-\delta)A_{t-1}+mI(n_{t-1})\right]mn'(z)-C_z(m,z;\kappa)=0 \\
\frac{\partial V}{\partial m}&\colon r I(n_{t-1})n(z)-C_m(m,z;\kappa)=0
\end{split}
\end{equation}

I will now use the implicit function theorem to interpret some
comparative statics in terms of the parameters of the model. I summarize
the results below

\(X\)

\(\partial m/\partial X\)

\(\partial z/\partial X\)

\(r\)

\(-\)

\(-\)

\(\delta\)

\(0\)

\(+\)

\(A_{t-1}\)

\(0\)

\(-\)

\(n_{t-1}\)

\(-\)

\(-\)

\(\kappa\)

\(-\)

\(-\)

Signs of first order conditions for the one-period model

The first column gives the relationships between increasing the
parameters and the optimal choice of quantity, \(m\). Producing more
quantity incurs costs that the artist substitutes away from when given
more slack by the other parameters. Interestingly, the previous period's
audience makes no impact on the artist's choice to produce a greater
quantity. This comes from the fact that prior audience is sunk in \(m\).
Since \(m\) can only influence the number of new listeners, the artist
need not consider how much of the previous audience they've retained.

Since \(z\) does not affect the future engagement measures (because I
only consider one period), the artist will try to substitute away from
using \(z\) in production because it is more costly. As such, when the
royalty rate increases, the artist will be able to recoup the same
amount of revenue selling fewer units, so they will chose to lower their
\(z\) due to it's increasing costliness.

One Period-Binary Choice Example

I will now explore the case in which the artist can only choose from one
of two options. In order to collapse the problem to a binary choice set,
instead of using a cost function, I will add an additional budget
constraint, \(C(m,z)=Y\) which implicitly defines \(z\) in terms of
\(m\). As such, let's consider the case where the artist is choosing to
produce either \(1\) or \(2\) products. If the artist chooses \(m=1\),
they can produce one higher quality good at quality \(\overline{z}\). In
the other case, the artist can produce \(2\) goods, each at quality
\(\underline{z}\). The artist will then choose:

\begin{equation} \label{eq:binary_choice}
\max\left\{r\left((1-\delta)A_{t-1}+I(n_{t-1})\right)n(\overline{z}),r\left((1-\delta)A_{t-1}+2I(n_{t-1})\right)n(\underline{z}) \right\}
\end{equation}

The artist's optimal production choice can be summarized as

\begin{equation} \label{eq:one_period_behavior}
\begin{cases}
\underline{z} &  n(\overline z) < \left(1+\frac{I_0}{I_0+(1-\delta)A_0}\right)n(\underline z)\\
\textrm{Either} & n(\overline z) = \left(1+\frac{I_0}{I_0+(1-\delta)A_0}\right)n(\underline z) \\
\overline{z}, &  n(\overline z) > \left(1+\frac{I_0}{I_0+(1-\delta)A_0}\right)n(\underline z)
\end{cases}
\end{equation}

By monotonicity of demand \(n(\overline{z})/n(\underline{z})>1\), so we
can think of \(\frac{I_0}{I_0+(1-\delta)A_0}\) as the minimal premium
for which the artist will produce the high quality option. Begin by
noting that the denominator in the above expression is the expected
audience size when the artist chooses the high quality option. As such,
we can interpret the premium as the percent of expected audience that is
earned by the algorithm. Holding other factors equal, an artist with a
larger initial audience, \(A_0\), is more likely to have a smaller
premium to induce high quality production relative to a new artist where
most of their audience is new. Therefore, established artists are more
likely to produce high quality content. In contrast, new artists are
likely to produce a greater number of lower-quality art and rely on the
algorithm to get their art into the hands of new consumers.

One Period with Talent

Since the talent parameter \(\kappa\) enters through the cost function,
it is not represented in the analysis above. I will now modify the model
slightly, allowing \(\kappa\) to enter into the binary choice problem.
Instead of constraining the artist to produce only one or two items at
qualities \(\underline{z}\) or \(\overline{z}\), I will now examine the
case in which the artist still chooses to produce an items of qualities
\(\underline{z}\) or \(\overline{z}\), however the artist's talent
dictates the number of releases they can produce. In particular, let
\(\underline{m}(\kappa)\) be the number of low-quality releases that an
artist of talent \(\kappa\) can produce in one period. Similarly, define
\(\overline{m}(\kappa)\) be the number of high quality releases the same
artist could produce in one period. In the eyes of a given artist (who
only has one talent), they only have two production options, however
artists with differing talents face different (though still binary)
production options.

The functions \(\underline m(\kappa),\overline m(\kappa)\) should mirror
reality in their properties. Namely should require that for all
\(\kappa\), \(\underline m(\kappa)>\overline m(\kappa)\), so more low
quality products can be produced than high quality ones. Furthermore,
both \(\underline m(\kappa)\) and \(\overline m(\kappa)\) should be
increasing in \(\kappa\) and should have diminishing marginal product,
\(\overline m''(\kappa),\underline m''(\kappa)<0\).

We can formulate the artist's one-period maximization problem as:

\begin{equation} \label{eq:binary_choice_talent}
\max\left\{r\left((1-\delta)A_{t-1}+\overline m(\kappa)I(n_{t-1})\right)n(\overline{z}),r\left((1-\delta)A_{t-1}+\underline m(\kappa)I(n_{t-1})\right)n(\underline{z}) \right\}
\end{equation}

With optimal choice of quality satisfying:

\begin{equation} \label{eq:one_period_talent_behavior}
\begin{cases}
\underline{z} &  n(\overline z) < \left(1+\frac{(\underline{m}(\kappa)-\overline{m}(\kappa))I_0}{\overline{m}(\kappa)I_0+(1-\delta)A_0}\right)n(\underline z)\\
\textrm{Either} & n(\overline z) = \left(1+\frac{(\underline{m}(\kappa)-\overline{m}(\kappa))I_0}{\overline{m}(\kappa)I_0+(1-\delta)A_0}\right)n(\underline z) \\
\overline{z}, &  n(\overline z) > \left(1+\frac{(\underline{m}(\kappa)-\overline{m}(\kappa))I_0}{\overline{m}(\kappa)I_0+(1-\delta)A_0}\right)n(\underline z)
\end{cases}
\end{equation}

\begin{prop}
Increases to talent incentivize an artist to produce low quality work whenever their talent satisfies $\underline{m}'(\kappa)\overline{m}(\kappa)-\overline{m}'(\kappa)\underline{m}(\kappa)\geq 0$.
\end{prop}

\begin{proof}
By \ref{eq:one_period_talent_behavior} the low quality strategy becomes more desireable when the following quantity increases:  $$\left(1+\frac{(\underline{m}(\kappa)-\overline{m}(\kappa))I_0}{\overline{m}(\kappa)I_0+(1-\delta)A_0}\right)$$

Thererfore, the low quality strategy gains desireability from increases to $\kappa$ whenever the derivative of the above expression is greater than zero. The $\kappa$ derivative is:
$$
\frac{(\underline{m}'(\kappa)-\overline{m}'(\kappa))I_0\left(\overline{m}(\kappa)I_0+(1-\delta)A_0\right)-(\underline{m}(\kappa)-\overline{m}(\kappa))I_0\left(\overline{m}'(\kappa)I_0\right)}{\left(\overline{m}(\kappa)I_0+(1-\delta)A_0\right)^2}
$$
The denominator is always positive, so positivity on the derivative only requires the numerator to also be positive. Expanding and simplify that expression gives the following inequality:
$$
(\underline{m}'-\overline{m}')(1-\delta)A_0+(\underline m'\overline m-\underline m\overline{m'})I_0>0
$$

Now suppose that $(\underline m'\overline m-\underline m\overline{m'})>0$, and by construction $\underline{m}>\overline{m}$.. Note that for an arbitrary $a,b,x,y\in \mathbb{R}^+$, when $ax-by>0$ $x>\frac{b}{a}y$. The condition $x>y$ is weaker than and is implied $x>\frac{b}{a}y$ whenever $a>b$. Letting $a=\overline{m},b=\underline{m},x=\underline m',y=\overline m '$, shows that both terms in the above inequality must be positive, so the whole derivative must also be positive.
\end{proof}

This sufficient condition
\(\underline{m}'(\kappa)\overline{m}(\kappa)-\overline{m}'(\kappa)\underline{m}(\kappa)\geq 0\),
can is equivalent to: talent causing a \(1\%\) increase in the number of
high quality units available for production results in a greater than
\(1\%\) increase in the number of low quality goods available for
production.

This says that increases to talent push an artist towards low quality
production whenever low quality production is cheap. We should not
innately have a sense for whether or not
\(\underline{m}'(\kappa)\overline{m}(\kappa)-\overline{m}'(\kappa)\underline{m}(\kappa)\geq 0\)
holds in reality. One could argue that an additional percent of low
quality production may be harder because you produce more low quality
goods, so additional units may be more difficult. Alternately, high
quality production may be harder because these goods are inherently
harder to produce.

Two Period Maximization

I now explore the case where the artist has the same options as in the
one-period no-cost case, presented in \ref{slug}. The artist can choose
one or two products at quality \(\overline z\) and \(\underline z\)
respectively. However, in this case, I examine the artist's behavior
over two periods. In this case, their choice of \(z\) in the first
period influences the number over impressions that the algorithm
produces next period. I will now introduce some new notation. Let
\(\overline {I}\) be the impressions awarded to the artist when their
choice of quality \(\overline z\) induces consumption
\(n(\overline z)\). Define \(\underline I\) analogously. I define
discount factor \(\beta\) that deflates the value of profits in
subsequent periods.

The artist has four options, either quality in either period. I will
denote the items in the strategy set:
\(\left\{\underline z\to\underline z,\overline z\to\underline z,\underline z\to\overline z,\overline z\to\overline z\right\}\),
where \(\underline z\to \overline{z}\) is the strategy with low quality
in period one, and high quality in period two and so on. (HOW AWFUL IS
THAT NOTATION?) Each strategy has the following utility.

\begin{equation} \label{eq:two_period_choices}
\begin{aligned} 
E[V(\underline z\to\underline z)]&=r\left((1-\delta)A_{0}+2I_0\right)n(\underline{z})+\beta\left(r\left((1-\delta)A_{1}+2I(n(\underline{z})\right)n(\underline{z})\right)\\
E[V(\overline z\to\underline z)]&=r\left((1-\delta)A_{0}+I_0\right)n(\overline{z})+\beta\left(r\left((1-\delta)A_{1}+2I(n(\overline{z})\right)n(\underline{z})\right)\\
E[V(\underline z\to\overline z)]&=r\left((1-\delta)A_{0}+2I_0\right)n(\underline{z})+\beta\left(r\left((1-\delta)A_{1}+I(n(\underline{z})\right)n(\overline{z})\right)\\
E[V(\overline z\to\overline z)]&=r\left((1-\delta)A_{0}+I_0\right)n(\overline{z})+\beta\left(r\left((1-\delta)A_{1}+I(n(\overline{z})\right)n(\overline{z})\right)
\end{aligned}
\end{equation}

Numerical Simulation of Two Period Optimal Behavior

The optimal strategy is the maximum of the four strategies listed in
\ref{eq:two_period_choices}, but the conditions under which certain
strategies are preferred are hard to characterize. I will use numerical
simulation to describe the conditions that produce certain optimal
strategies.

In order to simulate the possible conditions, I generate sample values
for each of the variables in \ref{eq:two_period_choices}. Absent a more
informed prior, I uniformly sample each variable in a reasonable range
of values. I summarize these ranges below:

Variable

Symbol

Minimal Possible Value

Maximal Possible Value

Streams per capita, high quality

\(n(\overline z)\)

\(0\)

\(10\)

Streams per capita, low quality

\(n(\underline z)\)

\(0\)

\(10^*\)

Starting audience

\(A_0\)

\(0\)

\(100\)

Discount factor

\(\beta\)

\(0\)

\(1\)

Audience depreciation rate

\(\delta\)

\(0\)

\(1\)

Expected impressions per song, high quality

\(I(n(\overline z))\)

\(0\)

\(100\)

Expected impression per song, low quality

\(I(n(\underline z))\)

\(0\)

\(100^*\)

Initial expected impressions

\(I_0\)

\(0\)

\(100^*\)

Sample range for simulated parameter values. \(^*\)Indicates that the
maximal value is limited by the high quality alternative

The values above are primarily aimed at describing the behavior of new
artists, where the algorithm will not show their content to more than
\(100\) new consumers per release. Furthermore, this assumes that
consumers will stream an artist's portfolio no more than \(10\) times,
an assumption that reflects more durable entertainment like a YouTube
video instead of a song. Additionally, I filter out any realization of
the random variable in which \(I_0\) or \(I(n(\underline z))\) are
greater than \(I(n(\overline z))\) because these are cases in which the
algorithm rewards new and low quality art more than high quality art. If
the algorithm is to be meaningful, then it should prioritize high
quality content. Similarly, I filter out any cases when
\(n(\underline{z})\geq n(\overline{z})\) because high quality work
should be streamed more than low quality alternatives.

With that, I generate a simulation dataset consisting of a realization
of each of the eight variables in the table above. I use \(9908\)
samples from the eight-dimensional parameter space. For each of these
sets of parameters, I compute the discounted two-period profit from each
of the four strategies. I then rank each of the strategies according to
their discounted profit to determine which strategy the agent will
choose. The following table computes the median value of each of the
parameters conditional on a given strategy being optimal.

\{r CleanSimulationData\} d \textless-
read.csv(``../inputs/SimulationData.csv'')

library(ggplot2) library(dplyr)

d \textless- d \%\textgreater\% mutate(premium=nu/nd-1)

d\_sum \textless- d \%\textgreater\% group\_by(winner) \%\textgreater\%
summarise(across(c(nu:I0,premium),median))

d\_sum \textless- d\_sum \%\textgreater\% select(-c(winner,premium))

rownames(d\_sum) \textless-
c(``\(\\underline{z}\\to\\underline{z}\),''``\(\\overline{z}\\to\\underline{z}\),''``\(\\underline{z}\\to\\overline{z}\),''``\(\\overline{z}\\to\\overline{z}\)'')

knitr::kable(d\_sum, digits=1, col.names =
c(``\(n(\\overline{z})\),''``\(n(\\underline{z})\),''``\(A_0\),''``\(\\beta\),''``\(\\delta\),''``\(I(n(\\overline{z}))\),''``\(I(n(\\underline{z}))\),''``\(I_0\)''),caption=``Median
value of sampled parameters conditional on a given strategy being
optimal,'' row.names = TRUE,escape=F,label=``MedianSim'')

There are many important insights that we can draw from the table above.
To do so, we can compare each of the values in the table to the
unconditional median of that parameter. For example, the overall median
of \(\delta\) is approximately \(.5\), but in the \(\delta\) column of
the \(\underline{z}\to\overline{z}\) row, the median value of \(\delta\)
is \(0.30\), much lower than the overall median. This means that when
the low quality, then high quality strategy is optimal, we expect there
to be relatively low audience depreciation relative to when other
strategies are optimal.

I will now briefly characterize the profile of parameters that produces
each of the winning outcomes. First, if
\(\underline{z}\to\underline{z}\) is optimal, then there is probably,
relatively little difference between the number of streams per capita of
high and low quality art. Additionally, there is relatively little
expected algorithmic benefit from high quality products. As
\(\underline{z}\to\underline{z}\) is the most pessimistic of the
possible strategies, it is not surprising that this strategy arises when
both the consumer and the algorithm are not very discerning.

For \(\overline{z}\to\underline{z}\) to be optimal, then consumers
should not be very picky, but the algorithm should. The algorithm should
make it hard to new artists to gain exposure (small \(I_0\)).
Additionally, the artist should have a stronger-than-average concern for
the future \(\beta=.6\), and a relatively high audience depreciation,
\(\delta=.6\). This description is consistent with intuition. If
consumers are not very picky between high and low quality, but initial
impressions are hard to come by, the artist will prioritize using more
releases to grow their audience instead of relying on the algorithm.

The \(\underline{z}\to\overline{z}\) is very rare among the points
sampled, and it requires a unique mix of parameters to be optimal.
Consumers should be moderately discerning, and the algorithm should
punish low quality content, however, the algorithm also needs to favor
new artists (\(I_0\)) more than low quality content
\(I(n(\underline z))\). Additionally greater present bias, and high
audience retention benefit this strategy. This result, albeit rare, has
important implications for how a streaming platform should treat its new
artists. If the platform wishes to incentivize future high quality
production from artists who start low quality, they should
algorithmically favor new content.

The final case \(\overline{z}\to\overline{z}\), is essentially the
opposite of \(\underline{z}\to\underline{z}\). For the high-quality only
strategy to exist, both consumers and the algorithm should be very
discerning of high quality content, those two conditions alone are
usually sufficient to result in \(\overline{z}\to\overline{z}\) as the
optimal strategy as evidenced by the other variables being near their
medians. This again has important implications for the streaming
platform, if they wish to establish an environment where high quality
content thrives, they should ensure that their algorithm rewards high
quality content, but the platform also relies on its consumers to be
equally discerning.

Discussion of the table above that the discerningness of the consumer
and the algorithm are important predictor of an artist's optimal
behavior. As crude measures of discerningness, let
\(\nu=n(\overline{z})/n(\overline{z})-1\) and
\(\iota=I(n(\overline{z})I/I(n(\overline{z}))-1\). These variables are
the percentage difference between the high and low quality versions of
streams per capita and expected impressions per release. Using these two

\{r DiscernPlot,fig.cap=``\textbackslash label\{fig:DiscernPlot\}
Consumer and Algorithmic Discerningness on Optimal Strategy''\} d
\textless- d \%\textgreater\% mutate(con\_dis=nu/nd-1, alg\_dis=Iu/Id-1)

cols \textless- c(``A''=``\#211030,''
``B''=``\#255059,''``C''=``\#FFBA3D,''``D''=``\#8C2730'') strats
\textless- c(expression(underline(z) \%-\textgreater\% underline(z)),
expression(bar(z) \%-\textgreater\%
underline(z)),expression(underline(z) \%-\textgreater\%
bar(z)),expression(bar(z) \%-\textgreater\% bar(z)))

ggplot(d)+
geom\_point(aes(x=log(con\_dis),y=log(alg\_dis),color=winner),alpha=.1)+
scale\_color\_manual(values = cols, labels = unname(strats) )+
theme\_bw()+ guides(colour = guide\_legend(override.aes = list(alpha =
1)))+ theme(panel.grid.major = element\_blank(), panel.grid.minor =
element\_blank(), panel.background = element\_blank(), axis.line =
element\_line(colour =
``black''))+theme(text=element\_text(family=``serif''),legend.title =
element\_blank())+ xlab(TeX(``Log Consumer Discerningness,
\(\\ln(\\iota)\)''))+ ylab(TeX(``Log Algorithm Discerningness,
\(\\ln(\\nu)\)''))

(I KNOW THE LEGEND IS MESSED UP THERE, BUT I'LL FIX IT LATER BECAUSE I
WANT TO GET TO THE INSIGHTS)

Figure 2 (FIX LINK) shows that while there is some overlap between the
optimal strategies, the regions are mostly distinct and contiguous. For
example, whenever \(\ln(\nu)>0\) (high consumer discerningness, almost
no other parameters matter and we are essentially guaranteed that the
\(\overline{z}\to\overline{z}\) is the ideal strategy. On the contrary
when \(\ln(\nu),\ln(\iota)<0\) the ideal strategy is almost always
\(\underline{z}\to\underline{z}\). It's really quadrant II where
\(\ln(\iota)>0\), but \(\ln(\nu)<0\), that the ideal strategy is least
clear. Since high algorithmic discernment, and low consumer discernment
push in opposite directions, it makes sense that quadrant II is
ambiguous. What is less clear is why quadrant IV is not equally
ambiguous. This seems to indicate that high consumer discernment
outweighs low algorithmic discernment, but the inverse is not always
true.

Random reputational shocks

The preceding sections rely on the content creator laying out a strategy
based on their expectations of the two period ahead. However, the model
in this paper is capable of replicating the unpredictability of the
algorithm. In particular, in each period, the algorithm adds a series of
unforecastable shock \(\varepsilon_i\) to the audience size evidenced in
\ref{eq:eqn_of_motion}.

In this section, I will allow the content creator to change their
strategy before period two (after having seen the effect of one random
shock). This will allow to explore a few relevant aspects of a real
content creator's decision making. First I will explore whether or not
certain ex ante strategies are more susceptible to change after shocks
occur.

Additionally, I will be able to explore whether or not mean zero shocks
can decrease overall content creator welfare by forcing people to
reallocate their resources. For example, a creator might invest heavily
in producing high quality products, but the shock forces them to pursue
low quality alternatives in period two. Their utility from being forced
to switch may now be lower than if they had pursued the low quality
option in both periods.

I implement these algorithmic shock according to the following
procedure:

Use the same sample parameters as in the previous section.

Compute the expected utility maximizing strategy as before.

With this expected utility maximizing choice, infer the first period
choice at each set of parameters.

Compute the expected audience size and expected utility for the high and
low quality strategies in the second period.

Compute the appropriate number of random shocks based on the number of
products in period one.

Add the random shocks to the deterministic audience size from period
one.

Compute new expected utilities for high and low strategies with the
shock from period one.

It is important to note that the content creator is still choosing an
expected utility because they cannot forecast the size of the shock in
period two. The random shocks that I add to production were normally
distributed with mean zero and standard deviation 100. The standard
deviation was calibrated so that roughly 10\% of simulations changed
their strategy.

\{r shock sense table and readin\} d \textless-
read.csv(``../Inputs/SimulationDataWithShocks.csv'')

library(dplyr)

ggplot(d \%\textgreater\% filter(different==``True''))+
geom\_violin(aes(x=winner,y=total\_shock,col=winner,fill=winner),alpha=.5)+
scale\_color\_manual(values = cols, labels = unname(strats) )+
scale\_fill\_manual(values = cols, labels = unname(strats) )+
theme\_bw()+ guides(colour = guide\_legend(override.aes = list(alpha =
1)))+ theme(panel.grid.major = element\_blank(), panel.grid.minor =
element\_blank(), panel.background = element\_blank(), axis.line =
element\_line(colour =
``black''))+theme(text=element\_text(family=``serif''),legend.title =
element\_blank())+ scale\_x\_discrete(labels=unname(strats))

\hypertarget{for_table--}{%
\section{for\_table \textless-}\label{for_table--}}

\hypertarget{d}{%
\section{d \%\textgreater\%}\label{d}}

\hypertarget{selectwinnerdifferent}{%
\section{select(winner,different)
\%\textgreater\%}\label{selectwinnerdifferent}}

\hypertarget{group_bywinner}{%
\section{group\_by(winner) \%\textgreater\%}\label{group_bywinner}}

\hypertarget{summarizenumber-of-simulationsndifferent}{%
\section{\texorpdfstring{summarize(\texttt{Number\ of\ Simulations}=n(),different)
\%\textgreater\%}{summarize(Number of Simulations=n(),different) \%\textgreater\%}}\label{summarizenumber-of-simulationsndifferent}}

\hypertarget{group_bywinnerdifferent}{%
\section{group\_by(winner,different)
\%\textgreater\%}\label{group_bywinnerdifferent}}

\hypertarget{summarizenumber-of-switchesnnumber-of-simulations}{%
\section{\texorpdfstring{summarize(\texttt{Number\ of\ Switches}=n(),\texttt{Number\ of\ Simulations})
\%\textgreater\%}{summarize(Number of Switches=n(),Number of Simulations) \%\textgreater\%}}\label{summarizenumber-of-switchesnnumber-of-simulations}}

\hypertarget{unique}{%
\section{unique() \%\textgreater\%}\label{unique}}

\hypertarget{filterdifferenttrue}{%
\section{filter(different==``True'')
\%\textgreater\%}\label{filterdifferenttrue}}

\hypertarget{ungroup}{%
\section{ungroup() \%\textgreater\%}\label{ungroup}}

\hypertarget{select-different}{%
\section{select(-different) \%\textgreater\%}\label{select-different}}

\hypertarget{renameex-ante-strategywinner}{%
\section{\texorpdfstring{rename(\texttt{Ex\ ante\ Strategy}=winner)
\%\textgreater\%}{rename(Ex ante Strategy=winner) \%\textgreater\%}}\label{renameex-ante-strategywinner}}

\hypertarget{mutatepercent-switched100number-of-switchesnumber-of-simulations}{%
\section{\texorpdfstring{mutate(\texttt{Percent\ Switched}=100*\texttt{Number\ of\ Switches}/\texttt{Number\ of\ Simulations})
\%\textgreater\%}{mutate(Percent Switched=100*Number of Switches/Number of Simulations) \%\textgreater\%}}\label{mutatepercent-switched100number-of-switchesnumber-of-simulations}}

\hypertarget{select-ex-ante-strategy}{%
\section{\texorpdfstring{select(-\texttt{Ex\ ante\ Strategy})}{select(-Ex ante Strategy)}}\label{select-ex-ante-strategy}}

\hypertarget{section}{%
\section{}\label{section}}

\hypertarget{rownamesfor_table--}{%
\section{rownames(for\_table) \textless-}\label{rownamesfor_table--}}

\hypertarget{cunderlineztounderlinezoverlineztounderlinezunderlineztooverlinezoverlineztooverlinez}{%
\section{\texorpdfstring{c(``\(\\underline{z}\\to\\underline{z}\),''``\(\\overline{z}\\to\\underline{z}\),''``\(\\underline{z}\\to\\overline{z}\),''``\(\\overline{z}\\to\\overline{z}\)'')}{c(``\textbackslash\textbackslash underline\{z\}\textbackslash\textbackslash to\textbackslash\textbackslash underline\{z\},''\,``\textbackslash\textbackslash overline\{z\}\textbackslash\textbackslash to\textbackslash\textbackslash underline\{z\},''\,``\textbackslash\textbackslash underline\{z\}\textbackslash\textbackslash to\textbackslash\textbackslash overline\{z\},''\,``\textbackslash\textbackslash overline\{z\}\textbackslash\textbackslash to\textbackslash\textbackslash overline\{z\}'')}}\label{cunderlineztounderlinezoverlineztounderlinezunderlineztooverlinezoverlineztooverlinez}}

\hypertarget{section-1}{%
\section{}\label{section-1}}

\hypertarget{knitrkablefor_table}{%
\section{knitr::kable(for\_table,}\label{knitrkablefor_table}}

\hypertarget{digits1captionfragility-of-various-strategies-under-algorithmic-shock}{%
\section{digits=1,caption=``Fragility of various strategies under
algorithmic
shock,''}\label{digits1captionfragility-of-various-strategies-under-algorithmic-shock}}

\hypertarget{row.names-trueescapeflabelmediansim}{%
\section{row.names =
TRUE,escape=F,label=``MedianSim'')}\label{row.names-trueescapeflabelmediansim}}

(MAKE THIS LOOK LESS AWFUL)

The table above reveals that the low-quality to high-quality strategy is
the most sensitive to perturbation. This result makes sense because

Made shock large enough so about 10\% switched strategies

If a person initially choosing a high quality period two, then a
positive shock will never make them choose to switch to low quality

Similarly, if a person was planning for a low quality period two, then a
negative shock never causes them to switch to high quality

Anecdotally, state \(A\), which is \(\underline{z}\to\underline{z}\) is
the least sensitive to shocks (even large ones). This is a notable
result because it's not like \(A\) is the most common result, that's
\(D\). But it's hard to say whether or not I've just built the
parameters so that the worst bad case is way worse than the best good
case

Next Steps, again

In that numerical simulation, see which strategy produces the biggest
audiences for a given set of parameters

Do the sensitivity analysis of how shocks affect behavior

Get a better structure for the results section (though that may only
come once I've narrowed down the results that are worth including).

\newpage

References

\hypertarget{refs}{}
\begin{CSLReferences}{1}{0}
\leavevmode\vadjust pre{\hypertarget{ref-aguiarQualityPredictabilityWelfare2018}{}}%
Aguiar, L., \& Waldfogel, J. (2018). Quality {Predictability} and the
{Welfare Benefits} from {New Products}: {Evidence} from the
{Digitization} of {Recorded Music}. \emph{Journal of Political Economy},
\emph{126}(2), 492--524. \url{https://doi.org/10.1086/696229}

\leavevmode\vadjust pre{\hypertarget{ref-andersonLongTailWhy2006}{}}%
Anderson, C. (2006). \emph{The long tail: Why the future of business is
selling less of more} (1st ed). {Hyperion}.

\leavevmode\vadjust pre{\hypertarget{ref-benderAttractingArtistsMusic2021}{}}%
Bender, M., Gal-Or, E., \& Geylani, T. (2021). Attracting artists to
music streaming platforms. \emph{European Journal of Operational
Research}, \emph{290}(3), 1083--1097.
\url{https://doi.org/10.1016/j.ejor.2020.08.049}

\leavevmode\vadjust pre{\hypertarget{ref-elberseShouldYouInvest2008}{}}%
Elberse, A. (2008). Should {You Invest} in the {Long Tail}?
\emph{Harvard Business Review}, 11.

\leavevmode\vadjust pre{\hypertarget{ref-hillerRiseStreamingMusic2017}{}}%
Hiller, R. S., \& Walter, J. M. (2017). The {Rise} of {Streaming Music}
and {Implications} for {Music Production}. \emph{Review of Network
Economics}, \emph{16}(4), 351--385.
\url{https://doi.org/10.1515/rne-2017-0064}

\leavevmode\vadjust pre{\hypertarget{ref-kleinRoleMarketForces1981}{}}%
Klein, B., \& Leffler, K. (1981). The {Role} of {Market Forces} in
{Assuring Contractual Performance}. \emph{The Journal of Political
Economy}.

\leavevmode\vadjust pre{\hypertarget{ref-macdonaldEconomicsRisingStars1988}{}}%
MacDonald, G. M. (1988). The {Economics} of {Rising Stars}. \emph{The
American Economic Review}, \emph{78}(1), 155--166.

\leavevmode\vadjust pre{\hypertarget{ref-rosenEconomicsSuperstars1981a}{}}%
Rosen, S. (1981). The {Economics} of {Superstars}. \emph{The American
Economic Review}, \emph{71}(5), 845--858.

\leavevmode\vadjust pre{\hypertarget{ref-shapiroPremiumsHighQuality1983}{}}%
Shapiro, C. (1983). Premiums for {High Quality Products} as {Returns} to
{Reputations}. \emph{The Quarterly Journal of Economics}, \emph{98}(4),
659--679. \url{https://doi.org/10.2307/1881782}

\leavevmode\vadjust pre{\hypertarget{ref-spenceJobMarketSignaling1973}{}}%
Spence, M. (1973). Job {Market Signaling}. \emph{The Quarterly Journal
of Economics}, \emph{87}(3), 355--374.
\url{https://doi.org/10.2307/1882010}

\leavevmode\vadjust pre{\hypertarget{ref-thomesEconomicAnalysisOnline2013}{}}%
Thomes, T. P. (2013). An economic analysis of online streaming music
services. \emph{Information Economics and Policy}, \emph{25}(2), 81--91.
\url{https://doi.org/10.1016/j.infoecopol.2013.04.001}

\leavevmode\vadjust pre{\hypertarget{ref-tomscottWhyYouTubeAlgorithm2017}{}}%
Tom Scott. (2017). \emph{Why {The YouTube Algorithm Will Always Be A
Mystery}}. \url{https://www.youtube.com/watch?v=BSpAWkQLlgM}.

\leavevmode\vadjust pre{\hypertarget{ref-youtubecreatorsWhatIdealVideo2018}{}}%
YouTube Creators. (2018). \emph{What's the {Ideal Video Length}?
\textbar{} {Master Class} \#1 ft. {Today I Found Out}}.
\url{https://www.youtube.com/watch?v=uB-1j2ZOjlw}.

\leavevmode\vadjust pre{\hypertarget{ref-YouTubeStarsRhett2020}{}}%
{YouTube} stars {Rhett} and {Link} explain how algorithm changes
supercharged their business to an estimated \$17.5 million in yearly
income {Indilens News Team} ! {Live Daily News On India} and {Around The
World}. (2020). In \emph{Indilens ! Live Daily News on India}.
\url{https://indilens.com/554159-youtube-stars-rhett-and-link-explain-how-algorithm-changes-supercharged-their-business-to-an-estimated-17-5-million-in-yearly-income/}.

\end{CSLReferences}

\end{document}
