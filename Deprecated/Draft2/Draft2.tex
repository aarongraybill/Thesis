% Options for packages loaded elsewhere
\PassOptionsToPackage{unicode}{hyperref}
\PassOptionsToPackage{hyphens}{url}
%
\documentclass[
]{article}
\usepackage{lmodern}
\usepackage{amssymb,amsmath}
\usepackage{ifxetex,ifluatex}
\ifnum 0\ifxetex 1\fi\ifluatex 1\fi=0 % if pdftex
  \usepackage[T1]{fontenc}
  \usepackage[utf8]{inputenc}
  \usepackage{textcomp} % provide euro and other symbols
\else % if luatex or xetex
  \usepackage{unicode-math}
  \defaultfontfeatures{Scale=MatchLowercase}
  \defaultfontfeatures[\rmfamily]{Ligatures=TeX,Scale=1}
\fi
% Use upquote if available, for straight quotes in verbatim environments
\IfFileExists{upquote.sty}{\usepackage{upquote}}{}
\IfFileExists{microtype.sty}{% use microtype if available
  \usepackage[]{microtype}
  \UseMicrotypeSet[protrusion]{basicmath} % disable protrusion for tt fonts
}{}
\makeatletter
\@ifundefined{KOMAClassName}{% if non-KOMA class
  \IfFileExists{parskip.sty}{%
    \usepackage{parskip}
  }{% else
    \setlength{\parindent}{0pt}
    \setlength{\parskip}{6pt plus 2pt minus 1pt}}
}{% if KOMA class
  \KOMAoptions{parskip=half}}
\makeatother
\usepackage{xcolor}
\IfFileExists{xurl.sty}{\usepackage{xurl}}{} % add URL line breaks if available
\IfFileExists{bookmark.sty}{\usepackage{bookmark}}{\usepackage{hyperref}}
\hypersetup{
  pdftitle={First Dynamic Programming Example},
  pdfauthor={Aaron Graybill},
  hidelinks,
  pdfcreator={LaTeX via pandoc}}
\urlstyle{same} % disable monospaced font for URLs
\usepackage[margin=1in]{geometry}
\usepackage{graphicx,grffile}
\makeatletter
\def\maxwidth{\ifdim\Gin@nat@width>\linewidth\linewidth\else\Gin@nat@width\fi}
\def\maxheight{\ifdim\Gin@nat@height>\textheight\textheight\else\Gin@nat@height\fi}
\makeatother
% Scale images if necessary, so that they will not overflow the page
% margins by default, and it is still possible to overwrite the defaults
% using explicit options in \includegraphics[width, height, ...]{}
\setkeys{Gin}{width=\maxwidth,height=\maxheight,keepaspectratio}
% Set default figure placement to htbp
\makeatletter
\def\fps@figure{htbp}
\makeatother
\setlength{\emergencystretch}{3em} % prevent overfull lines
\providecommand{\tightlist}{%
  \setlength{\itemsep}{0pt}\setlength{\parskip}{0pt}}
\setcounter{secnumdepth}{-\maxdimen} % remove section numbering

\title{First Dynamic Programming Example}
\author{Aaron Graybill}
\date{October 1, 2021}

\begin{document}
\maketitle

\hypertarget{introduction}{%
\subsection{Introduction}\label{introduction}}

In this preliminary draft, I use dynamic programming to model an
artist's optimal investment in reputation over time. Previous models
have looked for equilibria and market clearance, and this model will
explore a single artist's ideal tradeoff between investing in reputation
(given the market's reputation evolution function) and taking a less
costly option.

The analysis is nuanced by stochasticity in the equation of motion. The
artist can invest in their reputation, but this investment may not
payoff. The artist may have a ``bad gig'' caused by outside factors
which damages their reputation.

\hypertarget{the-model}{%
\subsection{The Model}\label{the-model}}

In period \(t\), denote the artist's reputation \(R_t\). The artist is
assumed to be a price taker, and that the price of their product is
solely determined by their reputation coming into period \(t\). We
denote this price function, \(p(R_t)\). The artist is assumed to sell
only one item per period. The artist's reputation is determined by the
quality of the products they have produced in all pervious period. The
quality that the artist produces is determined by a stochastic quality
production function \(q(z,\varepsilon)\) where the effort is given by
\(z\in\mathbb{R}\). In addition to effort, quality is determined by the
value of a random variable \(\varepsilon\). We assume that reputation is
exclusively determined by the history of realized values of quality,
\(q_t(z,\varepsilon)\). We assume that consumers update their reputation
according to \(R_{t+1}=f(q_t,R_t)\).

We assume that each unit of effort has some opportunity cost given by
\(c(z)\). We further assume that the artist maximizes their discounted
stream of profits. Denote the discount factor \(\beta\). An artist's
profit in time \(t\) is given by \(p(R_t)-c(z_t)\). However, the artist
knows that their choice of effort will affect their future reputation
and the prices they will receive as a result.

We can formulate the agent's discounted profits using the following
dynamic problem: \[
V(R_t)=\max_z\left\{p(R_t)-c(z_t)+\beta V(R_{t+1})\right\}\mid R_{t+1}=f(q(z_t,\varepsilon),R_t)
\]

\hypertarget{reproducing-shapiro}{%
\subsection{Reproducing Shapiro}\label{reproducing-shapiro}}

The reputational update function, \(f\), is central to this analysis,
and various functional forms should be explored. For now, assume that
\(R_{t}=q_{t-1}\), this mirrors the analysis conducted by (Shapiro,
1983), and I use this as a first pass to ensure that Shapiro's results
can be replicated. Shapiro's work includes a deterministic quality
function---the firm produces exactly the quality it wishes to---so I
begin by examining the case in which \(\varepsilon\) is the degenerate
random variable taking \(0\) with certainty.

At this point, we can let reputation and quality have units such that
\(p(R_t)=R_t\). In this case, the dynamic program can be rewritten as:

\[
V(R_t)=\max_z\left\{R_t-c(z_t)+\beta V(z_t)\right\}
\]

Of course, the solution to this dynamic program greatly depends on the
functional form of \(c(z)\). As a first pass, I will assume that
\(c(z)=\gamma z\), so there is constant marginal costs \(\gamma\)

\[
V(z_t)=\max_z\left\{R_t-\gamma z_t+\beta V(z_t)\right\}
\]

Suppose I conjecture that the true optimal effort supplied in every
period is given by \(z_{t+1}=\theta z_t\), where \(\theta\) is a
currently undetermined, time-invariant fraction of the previous effort.

Instead of the Bellman Equation we can alternately view the problem as
maximizing the lifetime discounted sum of profits:

\[
\sum_{t=0}^\infty \beta^t\left( R_t-\gamma z_t\right)
\]

Assuming that the agent always supplies the same fraction of their
previous utility, \(\theta\), we can rewrite the expression as:

\[
\begin{align*}
&R_0-\gamma z_0+\sum_{t=1}^\infty \beta^t\left( \theta^tz_0-\gamma \theta^tz_0\right)\\
&R_0-\gamma z_0+z_0(1-\gamma)\sum_{t=1}^\infty \left(\beta\theta\right)^t\\
&R_0-\gamma z_0+z_0(1-\gamma)\frac{\beta\theta}{1-\beta\theta}\\
\end{align*}
\]

The agent then chooses \(\theta\) and \(z_0\) to maxize the expression
above. We have:

\[
\begin{align*}
\frac{\partial}{\partial\theta}&\colon z_0(1-\gamma)\frac{\beta(1-\beta\theta)-\beta\theta(-\beta)}{(1-\beta\theta)^2}=0\\
\frac{\partial}{\partial\theta}&\colon -\gamma+(1-\gamma)\frac{\beta\theta}{1-\beta\theta}=0
\end{align*}
\] \[
z_0(1-\gamma)\frac{\beta}{(1-\beta\theta)^2}=0\implies z_0=0
\]

\[
-\gamma+(1-\gamma)\frac{\beta\theta}{1-\beta\theta}=0\implies\frac{\gamma}{\beta(1-\gamma)}=\frac{\theta}{1-\beta\theta}
\]

\[
\frac{\gamma}{\beta(1-\gamma)}=\frac{\theta}{1-\beta\theta}\implies\frac{\gamma}{\beta(1-\gamma)}-\frac{\theta}{1-\gamma}=\theta
\]

\[
\frac{\gamma}{\beta(1-\gamma)}-\frac{\theta}{1-\gamma}=\theta\implies\theta=\left(1+\frac{1}{1-\gamma}\right)\frac{\gamma}{\beta(1-\gamma)}
\]

\hypertarget{outstanding-questions}{%
\subsection{Outstanding questions:}\label{outstanding-questions}}

\begin{itemize}
\tightlist
\item
  Is there any meaningful difference between reputation and price?
  Specifically, can give reputation in dollar units so that the new
  reputation is just the price that consumers are willing to pay for the
  good. I think I can do this!
\end{itemize}

\hypertarget{refs}{}
\leavevmode\hypertarget{ref-shapiroPremiumsHighQuality1983}{}%
Shapiro, C. (1983). Premiums for High Quality Products as Returns to
Reputations. \emph{The Quarterly Journal of Economics}, \emph{98}(4),
659--679. \url{https://doi.org/10.2307/1881782}

\end{document}
